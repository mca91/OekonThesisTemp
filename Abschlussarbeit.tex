\documentclass[11pt,]{article}
\usepackage{lmodern}
\usepackage{amssymb,amsmath}
\usepackage{ifxetex,ifluatex}
\usepackage{fixltx2e} % provides \textsubscript
\ifnum 0\ifxetex 1\fi\ifluatex 1\fi=0 % if pdftex
  \usepackage[T1]{fontenc}
  \usepackage[utf8]{inputenc}
\else % if luatex or xelatex
  \ifxetex
    \usepackage{mathspec}
    \usepackage{xltxtra,xunicode}
  \else
    \usepackage{fontspec}
  \fi
  \defaultfontfeatures{Mapping=tex-text,Scale=MatchLowercase}
  \newcommand{\euro}{€}
\fi
% use upquote if available, for straight quotes in verbatim environments
\IfFileExists{upquote.sty}{\usepackage{upquote}}{}
% use microtype if available
\IfFileExists{microtype.sty}{%
\usepackage{microtype}
\UseMicrotypeSet[protrusion]{basicmath} % disable protrusion for tt fonts
}{}
\usepackage[lmargin=5cm,rmargin=2.5cm,tmargin=2.5cm,bmargin=2.5cm]{geometry}
\usepackage{color}
\usepackage{fancyvrb}
\newcommand{\VerbBar}{|}
\newcommand{\VERB}{\Verb[commandchars=\\\{\}]}
\DefineVerbatimEnvironment{Highlighting}{Verbatim}{commandchars=\\\{\}}
% Add ',fontsize=\small' for more characters per line
\usepackage{framed}
\definecolor{shadecolor}{RGB}{248,248,248}
\newenvironment{Shaded}{\begin{snugshade}}{\end{snugshade}}
\newcommand{\AlertTok}[1]{\textcolor[rgb]{0.94,0.16,0.16}{#1}}
\newcommand{\AnnotationTok}[1]{\textcolor[rgb]{0.56,0.35,0.01}{\textbf{\textit{#1}}}}
\newcommand{\AttributeTok}[1]{\textcolor[rgb]{0.77,0.63,0.00}{#1}}
\newcommand{\BaseNTok}[1]{\textcolor[rgb]{0.00,0.00,0.81}{#1}}
\newcommand{\BuiltInTok}[1]{#1}
\newcommand{\CharTok}[1]{\textcolor[rgb]{0.31,0.60,0.02}{#1}}
\newcommand{\CommentTok}[1]{\textcolor[rgb]{0.56,0.35,0.01}{\textit{#1}}}
\newcommand{\CommentVarTok}[1]{\textcolor[rgb]{0.56,0.35,0.01}{\textbf{\textit{#1}}}}
\newcommand{\ConstantTok}[1]{\textcolor[rgb]{0.00,0.00,0.00}{#1}}
\newcommand{\ControlFlowTok}[1]{\textcolor[rgb]{0.13,0.29,0.53}{\textbf{#1}}}
\newcommand{\DataTypeTok}[1]{\textcolor[rgb]{0.13,0.29,0.53}{#1}}
\newcommand{\DecValTok}[1]{\textcolor[rgb]{0.00,0.00,0.81}{#1}}
\newcommand{\DocumentationTok}[1]{\textcolor[rgb]{0.56,0.35,0.01}{\textbf{\textit{#1}}}}
\newcommand{\ErrorTok}[1]{\textcolor[rgb]{0.64,0.00,0.00}{\textbf{#1}}}
\newcommand{\ExtensionTok}[1]{#1}
\newcommand{\FloatTok}[1]{\textcolor[rgb]{0.00,0.00,0.81}{#1}}
\newcommand{\FunctionTok}[1]{\textcolor[rgb]{0.00,0.00,0.00}{#1}}
\newcommand{\ImportTok}[1]{#1}
\newcommand{\InformationTok}[1]{\textcolor[rgb]{0.56,0.35,0.01}{\textbf{\textit{#1}}}}
\newcommand{\KeywordTok}[1]{\textcolor[rgb]{0.13,0.29,0.53}{\textbf{#1}}}
\newcommand{\NormalTok}[1]{#1}
\newcommand{\OperatorTok}[1]{\textcolor[rgb]{0.81,0.36,0.00}{\textbf{#1}}}
\newcommand{\OtherTok}[1]{\textcolor[rgb]{0.56,0.35,0.01}{#1}}
\newcommand{\PreprocessorTok}[1]{\textcolor[rgb]{0.56,0.35,0.01}{\textit{#1}}}
\newcommand{\RegionMarkerTok}[1]{#1}
\newcommand{\SpecialCharTok}[1]{\textcolor[rgb]{0.00,0.00,0.00}{#1}}
\newcommand{\SpecialStringTok}[1]{\textcolor[rgb]{0.31,0.60,0.02}{#1}}
\newcommand{\StringTok}[1]{\textcolor[rgb]{0.31,0.60,0.02}{#1}}
\newcommand{\VariableTok}[1]{\textcolor[rgb]{0.00,0.00,0.00}{#1}}
\newcommand{\VerbatimStringTok}[1]{\textcolor[rgb]{0.31,0.60,0.02}{#1}}
\newcommand{\WarningTok}[1]{\textcolor[rgb]{0.56,0.35,0.01}{\textbf{\textit{#1}}}}
\usepackage{graphicx}
\makeatletter
\def\maxwidth{\ifdim\Gin@nat@width>\linewidth\linewidth\else\Gin@nat@width\fi}
\def\maxheight{\ifdim\Gin@nat@height>\textheight\textheight\else\Gin@nat@height\fi}
\makeatother
% Scale images if necessary, so that they will not overflow the page
% margins by default, and it is still possible to overwrite the defaults
% using explicit options in \includegraphics[width, height, ...]{}
\setkeys{Gin}{width=\maxwidth,height=\maxheight,keepaspectratio}
\ifxetex
  \usepackage[setpagesize=false, % page size defined by xetex
              unicode=false, % unicode breaks when used with xetex
              xetex]{hyperref}
\else
  \usepackage[unicode=true]{hyperref}
\fi
\hypersetup{breaklinks=true,
            bookmarks=true,
            pdfauthor={Name},
            pdftitle={Titel der Arbeit},
            colorlinks=true,
            citecolor=blue,
            urlcolor=blue,
            linkcolor=magenta,
            pdfborder={0 0 0}}
\urlstyle{same}  % don't use monospace font for urls
\setlength{\parindent}{0pt}
\setlength{\parskip}{6pt plus 2pt minus 1pt}
\setlength{\emergencystretch}{3em}  % prevent overfull lines
\setcounter{secnumdepth}{0}

%%% Use protect on footnotes to avoid problems with footnotes in titles
\let\rmarkdownfootnote\footnote%
\def\footnote{\protect\rmarkdownfootnote}

%%% Change title format to be more compact
\usepackage{titling}

% Create subtitle command for use in maketitle
\newcommand{\subtitle}[1]{
  \posttitle{
    \begin{center}\large#1\end{center}
    }
}

\setlength{\droptitle}{-2em}
  \title{Titel der Arbeit}
  \pretitle{\vspace{\droptitle}\centering\huge}
  \posttitle{\par}
\subtitle{Untertitel}
  \author{Name}
  \preauthor{\centering\large\emph}
  \postauthor{\par}
  \predate{\centering\large\emph}
  \postdate{\par}
  \date{today}

\usepackage[ngerman]{babel}
\usepackage{setspace}
\onehalfspacing

\begin{document}

%\maketitle

\begin{titlepage}
  \noindent\begin{minipage}{0.6\textwidth}
	  Universität Duisburg-Essen\\
	  Fakultät für Wirtschaftswissensschaften\\
	  Lehrstuhl für Ökonometrie\\
  \end{minipage}
	\begin{minipage}{0.4\textwidth}
	  \begin{flushright}
  	  \vspace{-0.5cm}
  	  \includegraphics*[width=5cm]{Includes/duelogo.jpg} % Hochschullogo
	  \end{flushright}
	\end{minipage}
  \\
  \vspace{1.5cm}
  \begin{center}
  \huge{Titel der Arbeit}\\
  \vspace{.25cm}
  \Large{Untertitel}\\
  \vspace{0.5cm}
  \large{Typ der Arbeit}\\
  \vspace{1cm}
  \large{Vorgelegt der Fakultät für Wirtschaftswissenschaften der \\ Universität Duisburg-Essen}\\
  \vspace{0.75cm}
  \large{von:}\\
  \vspace{0.5cm}
  Name\\
  \end{center}
  \vspace{4cm}
  
  \noindent\begin{minipage}[t]{0.3\textwidth}
  Matrikelnummer:
  \end{minipage}
  \begin{minipage}[t]{0.7\textwidth}
  \hspace{1cm}Matrikelnummer
  \end{minipage}  
  
  \noindent\begin{minipage}[t]{0.3\textwidth}
  Studienfach:
  \end{minipage}
  \begin{minipage}[t]{0.7\textwidth}
  \hspace{1cm}Studiengang
  \end{minipage}   
 
  \noindent\begin{minipage}[t]{0.3\textwidth}
  Erstgutachter:
  \end{minipage}
  \begin{minipage}[t]{0.7\textwidth}
  \hspace{1cm}Prof.~Dr.~Christoph Hanck
  \end{minipage}
  
  \noindent\begin{minipage}[t]{0.3\textwidth}
  Zweitgutachter:
  \end{minipage}
  \begin{minipage}[t]{0.7\textwidth}
  \hspace{1cm}Prof.~Dr.~Andreas Behr
  \end{minipage} 
  
  \noindent\begin{minipage}[t]{0.3\textwidth}
  Semester:
  \end{minipage}
  \begin{minipage}[t]{0.7\textwidth}
  \hspace{1cm}6. Fachsemester
  \end{minipage} 
  
  \noindent\begin{minipage}[t]{0.3\textwidth}
  Vsl. Studienabschluss:
  \end{minipage}
  \begin{minipage}[t]{0.7\textwidth}
  \hspace{1cm}Sommersemester 2019
  \end{minipage} 
  
  \noindent\begin{minipage}[t]{0.3\textwidth}
  Abgabedatum:
  \end{minipage}
  \begin{minipage}[t]{0.7\textwidth}
  \hspace{1cm}Abgabedatum
  \end{minipage}
     
\end{titlepage}


\pagenumbering{Roman} 
{
\hypersetup{linkcolor=black}
\setcounter{tocdepth}{3}
\tableofcontents
}
\newpage
\listoftables
\newpage
\listoffigures
\newpage
\pagenumbering{arabic} 
\hypertarget{r-markdown}{%
\subsection{R Markdown}\label{r-markdown}}

This is an R Markdown document. Markdown is a simple formatting syntax
for authoring HTML, PDF, and MS Word documents. For more details on
using R Markdown see \url{http://rmarkdown.rstudio.com}.

When you click the \textbf{Knit} button a document will be generated
that includes both content as well as the output of any embedded R code
chunks within the document. You can embed an R code chunk like this:

\begin{Shaded}
\begin{Highlighting}[]
\KeywordTok{summary}\NormalTok{(cars)}
\end{Highlighting}
\end{Shaded}

\begin{verbatim}
##      speed           dist       
##  Min.   : 4.0   Min.   :  2.00  
##  1st Qu.:12.0   1st Qu.: 26.00  
##  Median :15.0   Median : 36.00  
##  Mean   :15.4   Mean   : 42.98  
##  3rd Qu.:19.0   3rd Qu.: 56.00  
##  Max.   :25.0   Max.   :120.00
\end{verbatim}

\hypertarget{including-plots}{%
\subsection{Including Plots}\label{including-plots}}

You can also embed plots, for example:

\begin{figure}
\centering
\includegraphics{Abschlussarbeit_files/figure-latex/pressure-1.pdf}
\caption{\label{fig:pressure} Pressure}
\end{figure}

Note that the \texttt{echo\ =\ FALSE} parameter was added to the code
chunk to prevent printing of the R code that generated the plot. You can
label the plot above by including the label
\texttt{\textbackslash{}\textbackslash{}label\{fig:pressure\}} in the
chunk argument \texttt{fig.cap}. A reference to the plot is then made as
follows:

Looking at Figure \ref{fig:pressure} makes me happy.

\hypertarget{citations}{%
\subsection{Citations}\label{citations}}

The a \emph{bibtex} bibliography can be used for citations. The file
name of the bibliography used in this template is
\texttt{references.bib}

You can cite a source like this: (Keil et al. 2012) or Keil et al.
(2012)

A cited source will be automatically added to the end of the document.

\pagebreak

\hypertarget{references}{%
\subsection*{References}\label{references}}
\addcontentsline{toc}{subsection}{References}

\hypertarget{refs}{}
\leavevmode\hypertarget{ref-Keil_2012}{}%
Keil, Petr, Jonathan Belmaker, Adam M. Wilson, Philip Unitt, and Walter
Jetz. 2012. ``Downscaling of Species Distribution Models: A Hierarchical
Approach.'' Edited by Robert Freckleton. \emph{Methods Ecol Evol} 4 (1).
Wiley-Blackwell: 82--94.
\url{https://doi.org/10.1111/j.2041-210x.2012.00264.x}.

\newpage
\textbf{Eidesstattliche Versicherung}

\bigskip

Ich versichere an Eides statt durch meine Unterschrift, dass ich die vorstehende Arbeit selbständig und ohne fremde Hilfe angefertigt und alle Stellen, die ich wörtlich oder annähernd wörtlich aus Veröffentlichungen entnommen habe, als solche kenntlich gemacht habe, mich auch keiner anderen als der angegebenen Literatur oder sonstiger Hilfsmittel bedient habe. Die Arbeit hat in dieser oder ähnlicher Form noch keiner anderen Prüfungsbehörde vorgelegen.

\vspace{1cm}
\rule{0pt}{2\baselineskip} %
\par\noindent\makebox[2.25in]{\indent Essen, den \hrulefill} \hfill\makebox[2.25in]{\hrulefill}%
\par\noindent\makebox[2.25in][l]{} \hfill\makebox[2.25in][c]{Name}%


\end{document}
